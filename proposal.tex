\documentclass[12pt]{article}
\usepackage{fullpage,enumitem,amsmath,amssymb,graphicx,hyperref}
\usepackage[margin=1in]{geometry} 

\begin{document}

\begin{center}
{\Large CS221 Fall 2016 Project Proposal: Scrabble AI}

\begin{tabular}{rl}
  Authors: & Colleen Josephson $\{$cajoseph$\}$ and Rebecca Greene $\{$greenest$\}$\\
\end{tabular}
\end{center}

\subsubsection*{Introduction}
We propose a Scrabble AI to play against human or alternate AI
agents. Scrabble is a turn-based zero-sum game where players spell valid
English words on a 15x15 game board. Like a crossword puzzle, besides
the initial move, all words must intersect with one or more words
already on the board.

Each turn consists of the player putting a word onto the board and
then tallying up the letters scores against any multipliers to get a
positive integer score for that move. In general, long words and words
with uncommon letters score highly. Once a player has run out of tiles
or cannot make a move, the game ends. Whichever player has the highest
cumulative score wins.

 For each turn, a set of game state will be input, and a valid
move will be output. See next page for a concrete example. 

%% \textbf{Input:} (gameBoard, letterSet, specialCases, opponentState)

%% \textbf{Output:} (word, score, locationOnBoard, orientation)\\


%% The gameBoard is a 15x15 board with the current board state, letterSet
%% is the set of 7 letters that the AI has avavailable to form a word
%% with, the specialCases will be indicators for special game state
%% (skipped turns or going twice) and the opponentState will contain the
%% AI's estimate of the opponent state so it can potentially place words
%% adversarially and undercut the opponent's ability to score.

\subsubsection*{Metrics}
Since this is a zero-sum game, the utility is formally $+\infty$ if
the AI wins, $-\infty$ if the AI loses. The value function
$V_{agent,opp}(s)$, the expected utility for the game at state s.

We will first explore using a minimax value function where we assume
the opponent is trying to minimize our score.

As a first-pass, the AI will try to output the highest scoring word
possible for that board during a turn.  Eventually we would like to
incorporate estimates of the opponents state and techniques to play
adversarially that will minimize the opponent's score while maximizing
our own. For example, if there are two potential locations for
high-scoring words, we place the word in the location that will
maximally impede the opponent in creating future high-scoring
words. In summary:

\textbf{Local metric:} score for a single move; \textbf{Global metric:} end-game score

\subsubsection*{Baseline and Oracle}

We created an oracle and a baseline to get a better sense of the upper
and lower bounds of possible performance on the project.

The oracle searches through the scrabble dictionary to make a list of
moves that are possible with any seven letters. This is the upper bound of
any move possible for any scrabble player, human or otherwise, because
the likelihood that a player would possesses the correct tiles for that
move is quite low (and sometimes impossible).

Our baseline is given 7 random tiles and and makes the
first possible move that it finds. Occasionally this method might
produce an optimal move, but that would be highly unlikely.

\newpage
\subsubsection*{Examples and Preliminary Data}

\begin{footnotesize}\textbf{Example Input:}\\
\end{foornotesize}
\begin{scriptsize}
\newenvironment{mono}{\ttfamily}{\par}
\begin{mono}
  \noindent
00:[' ', ' ', ' ', ' ', ' ', ' ', ' ', ' ', ' ', ' ', ' ', ' ', ' ', ' ', ' ']\\
01:[' ', ' ', ' ', ' ', ' ', ' ', ' ', ' ', ' ', ' ', ' ', ' ', ' ', ' ', ' ']\\
02:[' ', ' ', ' ', ' ', ' ', ' ', ' ', ' ', ' ', ' ', ' ', ' ', ' ', ' ', ' ']\\
03:[' ', ' ', ' ', ' ', ' ', ' ', ' ', ' ', ' ', ' ', ' ', ' ', ' ', ' ', ' ']\\
04:[' ', ' ', ' ', ' ', 'S', ' ', ' ', ' ', ' ', ' ', ' ', ' ', ' ', ' ', ' ']\\
05:[' ', ' ', ' ', ' ', 'T', ' ', ' ', ' ', ' ', ' ', ' ', ' ', ' ', ' ', ' ']\\
06:[' ', ' ', ' ', ' ', 'A', ' ', ' ', ' ', ' ', ' ', ' ', ' ', ' ', ' ', ' ']\\
07:[' ', ' ', ' ', 'U', 'N', 'I', 'V', 'E', 'R', 'S', 'I', 'T', 'Y', ' ', ' ']\\
08:[' ', ' ', ' ', ' ', 'F', ' ', ' ', ' ', ' ', ' ', ' ', ' ', ' ', ' ', ' ']\\
09:[' ', ' ', ' ', ' ', 'O', ' ', ' ', ' ', ' ', ' ', ' ', ' ', ' ', ' ', ' ']\\
10:[' ', ' ', ' ', ' ', 'R', ' ', ' ', ' ', ' ', ' ', ' ', ' ', ' ', ' ', ' ']\\
11:[' ', ' ', ' ', ' ', 'D', ' ', ' ', ' ', ' ', ' ', ' ', ' ', ' ', ' ', ' ']\\
12:[' ', ' ', ' ', ' ', ' ', ' ', ' ', ' ', ' ', ' ', ' ', ' ', ' ', ' ', ' ']\\
13:[' ', ' ', ' ', ' ', ' ', ' ', ' ', ' ', ' ', ' ', ' ', ' ', ' ', ' ', ' ']\\
14:[' ', ' ', ' ', ' ', ' ', ' ', ' ', ' ', ' ', ' ', ' ', ' ', ' ', ' ', ' ']\\
----|00| |01| |02| |03| |04| |05| |06| |07| |08| |09| |10| |11| |12| |13| |14|
\end{mono}
\end{scriptsize}
\begin{footnotesize}
\noindent
\textbf{Example output:} $\{$word:'PIZZAZZY', score: 49, startPoint: (10, 6), orientation: 'v'$\}$
\end{footnotesize}

We created 10 test scenarios for a simplified version of the game with no multiplier tiles, and asked the oracle and baseline to place a word. The oracle performed about 10x better than the baseline\footnote{Note: the oracle is slow, it takes about 90 seconds to find a move. Because crafting example boards by hand and running the algorithm are time consuming, we kept the test set limited to 10.}: \textbf{Average oracle score:} 36.3, \textbf{Average baseline score:} 3.3


%% BASELN: max word is ER with score 2 at (7, 0),v
%% word:'PIZZAZZY', score: 49, startPoint: (10, 6), orientation: 'v'$\}$
%% BASELN: max word is HOT with score 6 at (6, 0),v
%% ORACLE: max word is PAZAZZES with score 37 at (7, 0),v
%% BASELN: max word is ENORS with score 5 at (6, 7),v
%% ORACLE: max word is PAZAZZES with score 37 at (7, 6),v
%% BASELN: max word is ART with score 3 at (5, 1),v
%% ORACLE: max word is JAZZBO with score 33 at (10, 0),v
%% BASELN: max word is IS with score 2 at (5, 0),v
%% ORACLE: max word is PIZZAZZY with score 49 at (10, 6),v
%% BASELN: max word is MU with score 4 at (3, 0),v
%% ORACLE: max word is PIZZAZZY with score 49 at (10, 6),v
%% BASELN: max word is AW with score 3 at (7, 0),v
%% ORACLE: max word is PAZAZZES with score 37 at (7, 5),v
%% BASELN: max word is BO with score 4 at (3, 0),v
%% ORACLE: max word is BEZAZZ with score 35 at (3, 8),v
%% BASELN: max word is ET with score 2 at (3, 0),v
%% ORACLE: max word is PIZZAZZY with score 49 at (4, 3),v
%% BASELN: max word is LO with score 2 at (14, 0),v
%% ORACLE: max word is QUIZZING with score 37 at (13, 0),v

\subsubsection*{Approach}
Next we plan to implement the turn-based framework of the game and add
score multipliers. This should be straightforward.

Our approach at a brute-force search Oracle was relatively successful
at optimizing for a local move. However, it was quite slow, and didn't
have constraints on the letters available like a real player. To speed
things up we will investigate using alpha-beta pruning and search
heuristics like A*. We'd like to see the average move take 10s or
less.

Additionally, as mentioned earlier, we'd like to incorporate
strategizing against the opponent into our AI. This is a bit vague, we
expect we'll have a better idea of what to do after completing the
PacMan assignment.


\subsubsection*{Prior Work and Conclusion}
By this point, most popular board games have AIs, and
Scrabble is no exception. In 2002, Brian Sheppard created 'Maven',
which has won games against several world champions. Another is the 1988 
Appel-Jacobsen Scrabble AI, optimized for speed rather than score.

We realize that a Scrabble AI is not groundbreaking, but this project
should satisfy our goal of creating a substantial body of code that
demonstrates our understanding of introductory AI.

\begin{footnotesize}
\begin{itemize}
\item Maven: \url{http://www.sciencedirect.com/science/article/pii/S0004370201001667}
\item Appel-Jacobsen \url{http://repository.cmu.edu/cgi/viewcontent.cgi?article=2580&context=compsci}
%\item Appel-Jacobsen:\url{https://www.cs.cmu.edu/afs/cs/academic/class/15451-s06/www/lectures/scrabble.pdf}
\item A Recent (2012) Scrabble AI: \url{http://ceur-ws.org/Vol-860/paper16.pdf}
\item An open-source amateur AI: \url{https://www.scotthyoung.com/blog/2013/02/21/wordsmith/}
\end{itemize}
\end{footnotesize}


\end{document}
